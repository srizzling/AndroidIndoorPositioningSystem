\documentclass[12pt]{article}
\usepackage{setspace}
\usepackage[margin=1in]{geometry}


\setlength\parindent{0pt}
\begin{document}
\begin{titlepage}
\begin{center}
	\Huge
	\textbf{A comparison of Bluetooth-Based Indoor Localization Systems} \\
	\large    

	\vspace{0.5cm}
	By \textbf{Sriram Venkatesh}\\
	Victoria University of Wellington\\
	300236116 \\

	\vspace{0.9cm}
	\textbf{Abstract}
\end{center}
\doublespacing
	Lorem ipsum dolor...


\end{titlepage}

\doublespacing
\section*{Introduction}
The usage of localization systems has risen recently due to the widespread adoption of smartphones. Smartphones take advantage of the information that is accessible from the wireless sensors available on mobile devices, which allow indoor positioning systems to accurately locate users or objects. One of the most popular sensors on a mobile device is the GPS receiver. \cite{HABITS} \\

However, GPS does not provide acceptable accuracy due to its line of site infrastructure \cite{fusionmethod} which means it is unable to locate users or objects inside buildings or structures. Due to this fact, indoor position systems do not use satellites, instead these systems rely on nearby anchors or nodes, which either actively locate or passively provide contextual information for devices to get sensed. \\

Accurate indoor localization has the potential to transform the way people navigate indoors in a similar way that GPS transformed the way people navigate outdoors. The two most common techniques of localization is the use of cellular radio and Wi-Fi radio. Both of these methods use a significantly 

\section*{Bluetooth}


\section*{Comparative Parameters}


\section*{Background}


\section*{Critque}


\section*{Conclusions}

\bibliographystyle{ieeetr}
\bibliography{bib}
\nocite{*}

\end{document}

